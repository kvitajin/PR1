Program přijímá na vstupu matici zadanou po řádcích, a to buď z klávesnice, nebo souboru. Vstup z klávesnice je průběžně validován, a průběžně je plněno dvourozměrné dynamicky alokované pole. Při vstupu ze souboru se kontroly (validace) neprovádí. Nad maticí v paměti lze zavolat Gaussovu eliminační metodu (implementujte ji jako funkci, jejímž konstantním vstupem je ukazatel na vstupní pole, výstupem ukazatel na nově vytvořené pole v trojúhelníkovém tvaru s vypočtenými hodnotami nezávislých proměnných, pokud to hodnost rozšířené matice dovolí). Program umožňuje výstup do H\+T\+ML souboru, obsahující výslednou matici (použijte H\+T\+ML tagy, jinak se formátem nijak zvlášť nezabývejte), pod maticí budou hodnoty nezávislých proměnných. Dalším možným výstupem bude výpis matice a hodnoty jejího determinantu, pokud to hodnost matice dovolí. 